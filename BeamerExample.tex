% Copyright 2004 by Till Tantau <tantau@users.sourceforge.net>.
%
% In principle, this file can be redistributed and/or modified under
% the terms of the GNU Public License, version 2.
%
% However, this file is supposed to be a template to be modified
% for your own needs. For this reason, if you use this file as a
% template and not specifically distribute it as part of a another
% package/program, I grant the extra permission to freely copy and
% modify this file as you see fit and even to delete this copyright
% notice. 

% --------------------------------------------
% Por Patricio López Juri (pelopez2@uc.cl)
\documentclass{beamer}


%% -----------------------------------------------------
%% Python & código -------------------------------------
%% -----------------------------------------------------

% Source: https://github.com/bamos/beamer-snippets

\usepackage[T1]{fontenc}
\usepackage{parskip,graphics,tikz,multimedia,hyperref,ulem,multicol}
\usetikzlibrary{arrows,positioning,shapes,decorations.pathmorphing,snakes}
\setlength{\itemsep}{0pt}\setlength{\parskip}{0pt}\setlength{\parsep}{0pt}
\graphicspath{{./images/}}

\usepackage{listings,textcomp,color}
\lstset{language=Python,upquote=true,
  basicstyle=\ttfamily\tiny,numbers=left,
  numberstyle=\tiny,stepnumber=1,numbersep=5pt,
  backgroundcolor=\color{white},frame=single,tabsize=2,
  showspaces=false,showstringspaces=false,showtabs=false,
  breaklines=true,breakatwhitespace=true,escapeinside=||,
  keywordstyle=\color{blue!70},stringstyle=\color{green!70!black!70},
  commentstyle=\color{black!80}\it
}

\usebackgroundtemplate{
  \tikz[overlay,remember picture]
  \node[yshift=10mm,anchor=south east,inner sep=0pt]
    at (current page.south east) {
    % Add small logo here if desired.
    % \includegraphics[width=0.5in]{Images/python-logo.png}
  };
}

\tikzset{
  yn/.style={draw,thick,rounded corners,fill=yellow!20,inner sep=.3cm},
  bn/.style={draw,thick,rounded corners,fill=blue!05,inner sep=.3cm},
  on/.style={draw,thick,rounded corners,fill=orange!20,inner sep=.3cm},
  rn/.style={draw,thick,rounded corners,fill=red!20,inner sep=.3cm},
  greenn/.style={draw,thick,rounded corners,fill=green!20,inner sep=.3cm},
  grayn/.style={draw,thick,rounded corners,fill=gray!20,inner sep=.3cm},
  to/.style={
    ->,>=stealth',shorten >=1pt,semithick,font=\sffamily\footnotesize
  },
  from/.style={
    <-,>=stealth',shorten >=1pt,semithick,font=\sffamily\footnotesize
  },
  tofrom/.style={
    <->,>=stealth',shorten >=1pt,semithick,font=\sffamily\footnotesize
  },
  every node/.style={align=center},
  squig/.style={->,line join=round,decorate, decoration={zigzag,
    segment length=8,amplitude=2,post=lineto,post length=2pt}}
}

\newcommand{\uncheckedBox}{\ensuremath{\square}}
\newcommand{\checkedBox}{\ensuremath{\text{\rlap{\checkmark}}\square}}

\beamertemplatenavigationsymbolsempty

\expandafter\def\expandafter\insertshorttitle\expandafter{%
  \insertshorttitle\hfill%
  \insertframenumber\,/\,\inserttotalframenumber}

\usepackage{enumitem}
\setlist[1]{itemsep=5pt}
\setitemize{label=\usebeamerfont*{itemize item}%
  \usebeamercolor[fg]{itemize item}
  \usebeamertemplate{itemize item}}
  
  
%% -----------------------------------------------------
%% Lenguaje --------------------------------------------
%% -----------------------------------------------------

% \usepackage[T1]{fontenc} Está mas arriba.
\usepackage{selinput}
\SelectInputMappings{%
  aacute={á},
  ntilde={ñ},
  Euro={€}
}
\usepackage{babel}


%% -----------------------------------------------------
%% Tema ------------------------------------------------
%% -----------------------------------------------------

% Colores principales
\usetheme{CambridgeUS}

% Para las URL
\definecolor{links}{HTML}{2A1B81}
\hypersetup{colorlinks,linkcolor=,urlcolor=links}


%% -----------------------------------------------------
%% Main ------------------------------------------------
%% -----------------------------------------------------

% Título, debe ir obligatoriamente
\title{Introducción a Python}

% Subtítulo opcional
\subtitle{Lo que debemos saber de Introducción a la programación}

% Ayudantes
\author{Ayudante 1 \and Ayudante 2}

% Universidad
\institute[UC]
{
  Departmento de Ciencias de la Computación\\
  Pontificia Universidad Católica de Chile
}

% Dimensión tiempo espacio a mostrar.
\date{IIC2233, 2015-1}

% Metadada
\subject{Ayudantía de Programación Avanzada}


% If you have a file called "university-logo-filename.xxx", where xxx
% is a graphic format that can be processed by latex or pdflatex,
% resp., then you can add a logo as follows:

% \pgfdeclareimage[height=0.5cm]{university-logo}{university-logo-filename}
% \logo{\pgfuseimage{university-logo}}

% Delete this, if you do not want the table of contents to pop up at
% the beginning of each subsection:
% \AtBeginSubsection[]
% {
%   \begin{frame}<beamer>{Outline}
%     \tableofcontents[currentsection,currentsubsection]
%   \end{frame}
% }


%% -----------------------------------------------------
%% Incio del documento ---------------------------------
%% -----------------------------------------------------

\begin{document}

% Agregamos la página de inicio (la del título)
\begin{frame}
  \titlepage
\end{frame}

% Tabla de contenidos, tiene hipervínculos.
% Los índices son tomados del nombre que se le asigne a las \section.
% Los sub-índices provienen de las \subsections
\begin{frame}{Tabla de contenidos}
  \tableofcontents
  % You might wish to add the option [pausesections]
\end{frame}


%% **********************
%% Sección **************
%% **********************
\section{Python}

%% Sub-sección 
%% **********************
\subsection{Lenguaje}

\begin{frame}{Lenguaje}{¿Qué sabemos?}
  \begin{itemize}
  \item
    Es \alert{\textit{Open Source}}, es decir, su código es público y cualquiera puede verlo.
  \item
    Es \alert{Agnóstico de Sistema Operativo \textit{(OS)}}, o sea, no importa si es Windows, UNIX, etc.
  \item
    Es un lenguaje \alert{interpretado}.
    \begin{itemize}
    \item
      Requiere de un intérprete, eso es lo que instalaron cuando les pedimos que bajaran Python.
    \item
      Al contrario de otros lenguajes que son \alert{compilados}.
    \item
      Los intérpretes están escritos en otro lenguaje, distinto a Python. 
    \end{itemize}
  \end{itemize}
\end{frame}

%% Sub-sección 
%% **********************
\subsection{Historia}

\begin{frame}
  \begin{center}
    \includegraphics[height=6cm]{Images/python-logo.png}
    \\
    \pause % Pausa sirve para que los siguientes elementos aparezcan en la siguiente slide.
    Alguien lo habrá inventado, ¿no?
  \end{center}
\end{frame}

%% Sub-sub-sección 
%% ///////////////
\subsubsection{Creador}

\begin{frame}{Creador}{Guido van Rossum}
    \begin{columns}[T] % contents are top vertically aligned
    \begin{column}[T]{0.6\textwidth} % each column can also be its own environment
    \textbf{Guido van Rossum} es un científico de la computación, conocido por ser el autor del lenguaje de programación Python.
    \end{column}
    \begin{column}[T]{0.4\textwidth} % alternative top-align that's better for graphics
         \includegraphics[height=6cm]{Images/GuidoImage.jpg}
    \end{column}
    \end{columns}
\end{frame}


%% **********************
%% Sección **************
%% **********************
\section{Sintaxis}

\begin{frame}[fragile]{Sintaxis}

    Una clase de ejemplo:

    \begin{lstlisting}[language=Python,keywordstyle=\bf,stringstyle=\it]
# book.py
class Book:

    def __init__(self, title='Untitled'):
        self.title = title

    def abrir(self):
        print("Book opened: \"{}\"".format(self.title))

book = Book()
book.abrir()

  \end{lstlisting}
  
  \begin{block}{Valores por defecto}
  Recordemos que podemos dar la opción de que un valor tome cierto valor al no estar presente.
  \end{block}
  
\end{frame}


%% **********************
%% Sección **************
%% **********************
\section{PEP8}

\begin{frame}{PEP8}{Standard}

\begin{exampleblock}{}
  {\large ``Code is read much more often than it is written.''}
  \vskip5mm
  \hspace*\fill{\small--- Guido van Rossum}
\end{exampleblock}

\begin{block}{PEP8}
Guía de estilo para mantener la consistencia y leíbilidad. \\
\url{https://www.python.org/dev/peps/pep-0008/}
\end{block}
\begin{example}
¿Espacios vs Tabs? \\
Se deben usar 4 espacios para identar.
\end{example}
\end{frame}


%% **********************
%% Sumario **************
%% **********************
% Al ponerle un * a \section hace que no aparezca en el índice.
\section*{Sumario}

\begin{frame}{Sumario}
  \begin{itemize}
  \item
    Python es un \alert{lenguaje interpretado}.
  \item
    Se deben seguir ciertas convenciones al programar, tanto estructurales como del \textit{layout}.
  \end{itemize}
\end{frame}


%% **********************
%% Lecturas o material ***
%% **********************
% Icons: http://tex.stackexchange.com/questions/68080/beamer-bibliography-icon
\appendix
\section<presentation>*{\appendixname}
\subsection<presentation>*{Lecturas y materiales}

\begin{frame}[allowframebreaks]
  \frametitle<presentation>{Lecturas y materiales}
    
  \begin{thebibliography}{10}
    
  \setbeamertemplate{bibliography item}[online]
  \bibitem{Doc}
    Documentación Python3.x
    \newblock {\em \url{https://docs.python.org/3}}.
    \newblock 6 de marzo, 2015.
 
  \setbeamertemplate{bibliography item}[online]
  \bibitem{PEP8}
    Guía de estilo PEP8.
    \newblock {\em \url{https://www.python.org/dev/peps/pep-0008}}.
    \newblock 1 de agosto, 2013.
  \end{thebibliography}
\end{frame}

\end{document}
